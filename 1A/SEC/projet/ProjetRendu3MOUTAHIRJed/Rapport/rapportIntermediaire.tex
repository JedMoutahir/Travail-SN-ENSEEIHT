\documentclass[frenchb]{article}
\usepackage[T1]{fontenc}
\usepackage[latin1]{inputenc}
%Pour utilisation sous unix
%\usepackage[utf8]{inputenc}
%\usepackage[utf8x]{inputenc}
\usepackage{a4wide}
\usepackage{graphicx}
\usepackage{amssymb}
\usepackage{color}
\usepackage{babel}

\begin{document}

\begin{figure}[t]
\centering
\includegraphics[width=5cm]{inp_n7.png}
\end{figure}

\title{\vspace{4cm} \textbf{Projet minishell :\\Rapport Interm�diaire}}
\author{MOUTAHIR Jed}
\date{\vspace{7cm} D�partement Sciences du Num�rique - Premi�re ann�e \\
2021-2022 }

\maketitle

%\newpage
%\tableofcontents
%\listoffigures

\newpage

%%%%%%%%%%%%%%%%%%%%%%%%%%%%%%%%%%%%%%%%%%%%%%%%%%%%%%%%%%%%%%%%%
\section{Questions trait�es}
%%%%%%%%%%%%%%%%%%%%%%%%%%%%%%%%%%%%%%%%%%%%%%%%%%%%%%%%%%%%%%%%%%
Les questions de 1 � 8 ont �t� compl�tement trait�es.

%%%%%%%%%%%%%%%%%%%%%%%%%%%%%%%%%%%%%%%%%%%%%%%%%%%%%%%%%%%%%%%%%%%%%
\section{Choix de conception}
%%%%%%%%%%%%%%%%%%%%%%%%%%%%%%%%%%%%%%%%%%%%%%%%%%%%%%%%%%%%%%%%%%%%

\subsection{Question 2}
Cf. $Q2.pdf$.

\subsection{Question 3}
Pour r�pondre � cette demande ont utilise \textbf{wait}.

\subsection{Question 4}
Apr�s avoir v�rifi� les arguments en entr�e, pour \textbf{cd}, ont utilise \textbf{chdir()}.\\

On utilise un simple \textbf{exit()} pour \textbf{exit}.

\subsection{Question 5}
Si la commande doit �tre ex�cut�e en arri�re plan (d�tection du \textbf{$\&$}), alors, on ne bloque pas l'invit� de commande. A la place, on lance la commande et on affiche � nouveau l'invit� de commande.

\subsection{Question 6}
De mani�re � pouvoir g�rer les processus et le mettre � jour, j'ai opt� pour :

\begin{enumerate}
	\item Une structure qui r�sume les informations du processus.
	\item Une liste des processus.
	\item Une suite de m�thodes qui g�rent ces processus.
\end{enumerate}

\subsection{Question 7}
Le fonctionnement ainsi que les m�thodes n�cessaires � suspendre un processus ont d�j� �t� impl�ment�s en question 6. De ce fait, il suffit de rajouter un handler pour le signal \textbf{SIGSTOP} et masquer ce signal pour les fils du minishell.

\subsection{Question 8}
On fait de la m�me fa�on que pour la question 7 mais avec le signal \textbf{SIGINT}.

\end{document} 